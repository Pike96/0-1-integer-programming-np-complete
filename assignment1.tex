\documentclass[mathserif]{beamer}
\usepackage{amsmath}

\renewcommand{\qed}{\hfill\square}

\title{NP-completeness of 0-1 Integer Programming}
\author{Kurt McAlpine, 2004750}

\begin{document}

\frame{\titlepage}

\begin{frame}
\frametitle{Description}
So Integer Programming is an optimisation problem in the following form.

\begin{align*}
\text{maximise}\ c^Tx\\
\text{subject to}\ Ax \leq b,\\
x \leq 0,\\
x \in \mathbb{Z}^n
\end{align*}

where $c$ and $b$ are vectors and $A$ is a matrix.

\end{frame}

\begin{frame}
\frametitle{Description}
A special case of integer programming is zero-one integer programming:
\begin{align*}
\text{maximise}\ c^Tx\\
\text{subject to}\ Ax \leq b,\\
x \leq 0,\\
x \in \{0, 1\}^n
\end{align*}
\end{frame}

\begin{frame}
\frametitle{Plan}
Prove that 0-1 integer programming is NP-complete
\begin{itemize}
\item Prove that it is in NP
\item Reduce 3SAT to 0-1 integer programming
\end{itemize}
\end{frame}

\begin{frame}
\frametitle{Prove that it is in NP}
\textbf{Lemma:} 0-1 Integer Programming is in NP\\
\textbf{Proof:} An verification algorithm $V$ can be constructed that takes the input $A$, $x \in \mathbb{Z}^n$ and $b$
and checks that the condition $Ax \leq b$ holds.
\end{frame}

\begin{frame}
\frametitle{Reduce 3SAT to 0-1 integer programming}
\begin{align*}
\text{Let}\ \phi = C_1 \wedge C_2 \wedge \ldots \wedge C_n\
\end{align*}
\begin{center}
be a boolean formula in conjunctive normal form. Each $C_i$ is the disjunction of 3 literals.
\end{center}
\begin{align*}
y_{i1} \vee y_{i2} \vee y_{i3}
\end{align*}

\begin{center}
Each literal is either a variable $v$ or a negated variable $\neg v$.\\

Assume our variables are $v_1,\ \ldots\ ,\ v_n$.
\end{center}

\end{frame}

\begin{frame}
\frametitle{Reduce 3SAT to 0-1 integer programming}
We turn $\phi$ into an arithmetic formula $\Phi$ in the following way:
\begin{itemize}
\item Introduce an integer variable $z_i$ for every boolean variable $v_i$. The
integer variable takes the value of either $0$ or $1$.\\$0$ represents false and
$1$ represents true.
\item Replace $\wedge$ with $\times$, $\vee$ with $+$ and variables $v_j$ with
	$z_j$ and negated variables $\neg v_j$ with $1 - z_j$
\item $\Phi$ is the arithmetic formula obtained from this procedure
\end{itemize}
\end{frame}

\begin{frame}
\frametitle{Reduce 3SAT to 0-1 integer programming}
The Integer Program is as follows:
\begin{align*}
\Phi \geq 1\\
0 \leq z_i \leq 1\ \text{for all}\ i
\end{align*}

\textbf{Lemma:} $\phi$ is satisfiable iff the integer program is feasible\\
\textbf{Proof:} Assume that $\phi$ is satisfiable and let $V$ be a satisfying
assignment. Construct $Z$ by setting $Z_i = 1$ if $V_i$ is true and
$Z_i = 0$ if $V_i$ is false. Then $\Phi \geq 1$.\\
The other way around, if $Z$ is a feasible solution for the integer program,
each clause must have an integer value of at least one. Hence the corresponding
truth assignment satisfies $\phi$.\\
$\qed$
\end{frame}




\end{document}
