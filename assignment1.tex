\documentclass[mathserif]{beamer}
\usepackage{amsmath}

\renewcommand{\qed}{\hfill\square}

\begin{document}

\begin{frame}
So Integer Programming is an optimisation problem in the following form.

\begin{align*}
\text{maximize}\ c^Tx\\
\text{subject to}\ Ax \leq b,\\
x \leq 0,\\
x \in \mathbb{Z}^n
\end{align*}

\pause

A special case of integer programming is zero-one integer programming:
\begin{align*}
\text{maximize}\ c^Tx\\
\text{subject to}\ Ax \leq b,\\
x \leq 0,\\
x \in \{0, 1\}^n
\end{align*}
\end{frame}

Reduce SAT to 0-1 integer programming

\begin{align*}
\text{Let}\ \phi = C_1 \wedge C_2 \wedge \ldots \wedge C_n\
\end{align*}
be a formula in conjunctive normal form. Each $C_i$ is the disjunction of 3 literals

\begin{align*}
x_{i1} \vee x_{i2} \vee x_{i3}
\end{align*}

and each literal is either a vairable $v$ or a negated variable $\neg v$

Assume our variables are $v_1,\ \ldots\ ,\ v_m$

\begin{frame}
We turn $\phi$ into an arithmatic formula $\Phi$ as follows
\begin{itemize}
\item Introduce an integer variable $z_i$ for every boolean variable $v_i$. The integer variable is contrained to take values zero and one, zero represents false and one represents true
\item Replace $\wedge$ by $\times$, $\vee$ by $+$ and variables $v_j$ by $z_j$ and negated variables $\neg v$ by $1 - z_j$
\item $\Phi$ be the arithmatic formula obtained this way
\end{itemize}
\end{frame}

\begin{frame}
The ILP is
\begin{align*}
\Phi \leq 1\\
0 \leq z_i \leq 1\\
z_i \in \mathbb{Z}\ \text{for all}\ i
\end{align*}

Lemma 1: $\Phi$ is satisfiable iff the integer program is feasible
\end{frame}

\begin{frame}
Proof: Assume that $\phi$ is satisfiable and let $v^*$ be a satisfying
assignment. Construct $z^*$ by setting $z^*_i = 1$ iff $v^*_i$ is true. Then
$\Phi \leq 1$.
Conversly, if $z^*$ is a feasible solution for the integer program, each clause
must have an integer value of at least one. Hence the corresponding truth
asssignment satisfies $\phi$.\\
$\qed$

Theorem: 0-1 Integer Programming is NP complete\\

Proof: The Lemma above shows that 0-1 IP is NP-hard. 
\end{frame}



\end{document}
