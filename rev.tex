\documentclass[mathserif]{beamer}
\usepackage{amsmath}

\title{NP-completeness of 0-1 Integer Programming}
\author{Kurt McAlpine, 2004750}

\begin{document}

\frame{\titlepage}

\begin{frame}
\frametitle{Description}

Integer programming is defined as the following optimisation problem:

\begin{align}
	max \{ c^\top x : Ax \leq b, x \in \mathbb{Z}^n\}
\end{align}

Where $A$ is an $m \times n$ integer matrix for $m,n \in \mathbb{N}$, $b \in
\mathbb{Z}^m$, $c \in \mathbb{Z}^n$\\

0-1 integer programming is a special case of integer programming where $x \in
\{0,1\}^n$

\end{frame}

\begin{frame}
\frametitle{Description}

Reformulating the integer programming optimisation problem into a decision
problem so that I can give a proof of NP-completeness.\\

``Given $A$ and $b$ does there exist an $x$ such that the following condition
is satisfied $Ax \geq b$''

\end{frame}

\begin{frame}
\frametitle{0-1 Integer Programming is in NP}

To show 0-1 integer programming is in NP we describe a Turing machine M that
can verify an instance in polynomial time.\\

$M = $``On input $(A, b, x)$\\
1. multiply matrix $A$ with vector $x$\\
2. for $i \in (1,\ldots,m)$\\
3. ~ if $(Ax)_i > b_i$ reject\\
4. accept

\end{frame}


\begin{frame}
\frametitle{0-1 Integer Programming is in NP}

Time complexity of the verifier:\\
1. Multiply each component of a row of $A$ with each component of $x$ and sum
each value:\\
~ $n$ multiplications $O(n)$\\
~ $n - 1$ additions $O(n)$\\
~ $O(n) + O(n) = O(n)$\\
~ Doing this for $m$ rows of $A$, $m \cdot O(n) = O(m \cdot n )$\\

2. Compare component wise each element of $(Ax)$ with each element of $b$, $m$
comparisons is $O(m)$\\

Final time complexity is $O(m \cdot n) + O(m) = O(m \cdot n)$

0-1 integer programming $\in \mathit{NP}$ \qed

\end{frame}

\begin{frame}
\frametitle{3SAT reduction to Integer Programming}

Reduction of 3SAT with $n$ variables and $m$ clauses to an instance of integer
programming where $Ax \geq b$\\

Definition. 0-1 integer programming is feasible iff $Ax \geq b$ holds\\

3SAT is a 3cnf boolean formula\\

\begin{align*}
C_1 \wedge C_2 \wedge \ldots \wedge C_m
\end{align*}

Were $C_i$ is a disjunction of 3 literals

\begin{align*}
	y_{i1} \vee y_{i2} \vee y_{i3}
\end{align*}

a literal may be negated i.e. $\neg y_{i1}$

\end{frame}

\begin{frame}
\frametitle{3SAT reduction to Integer Programming}

Let the 3SAT formula $S$ have clauses $C_1,\ldots,C_m$ and variables
$v_1,\ldots,v_n$\\

If $S$ is satisfiable for each clause there has to be at least one boolean
variable $v_i$ that is true or a negated variable $\neq v_j$ that evaluates to
true. Each clause of $S$ can be represented as an inequality by replacing
boolean variables $v_i$ with integer variables $x_i$ and negated boolean
variables $\neg v_j$ with $1 - x_j$ and replacing disjunction with addition. An
example of a transformation is the following:

\begin{align*}
	y_{i1} \vee y_{i2} \vee y_{i3}\\
	x_{1} \vee x_{2} \vee x_{3} \geq 1\\
	\text{or}\\
	\neg y_{i1} \vee y_{i2} \vee y_{i3}\\
	1 - x_{1} \vee x_{2} \vee x_{3} \geq 1
\end{align*}

\end{frame}

\begin{frame}
\frametitle{3SAT reduction to Integer Programming}

If 3SAT is satisfiable 0-1 integer programming is feasible. Since there is a satisfiable assignment for $S$, at least one variable in each
clause must be true or some negated variable evaluates to true. So for each
clause there is an inequality that evaluates to true.

If integer programming is feasible 3SAT is satisfiable. If integer is feasible
each inequality holds, therefor there must be an integer variable equal to 1 or
some $1 - x_i$ is equal to 1, therefor each corresponding clause $C_i$ has at
least one boolean variable that is true or a negated variable that evaluates to
true.

This implies that 0-1 integer programming is NP-hard.

Therefor this 0-1 integer programming $\in \mathit{NP}$, 0-1 integer
programming is NP-complete. \qed

\end{frame}




\end{document}
